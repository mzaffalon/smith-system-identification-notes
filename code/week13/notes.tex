% Created 2023-12-21 Do. 12:01
% Intended LaTeX compiler: pdflatex
\documentclass[11pt]{article}
\usepackage[utf8]{inputenc}
\usepackage[T1]{fontenc}
%\usepackage{graphicx}
\usepackage{amsmath}
\usepackage{amssymb}
\usepackage{hyperref}
\usepackage{enumitem}
%\date{\today}
%\title{}
\hypersetup{
 pdfauthor={},
 pdftitle={},
 pdfkeywords={},
 pdfsubject={},
 pdfcreator={Emacs 29.1 (Org mode 9.6.6)},
 pdflang={English}}
\begin{document}


\section*{Problem 1}
\label{sec:prob1}

\begin{enumerate}
\item [\textbf{Part 1}]
  We work with the model
  \begin{equation*}
    y(k) = \frac{B(z)}{A(z)}u(k) + \frac{H_1(z)}{A(z)}e(k).
  \end{equation*}
  \begin{enumerate}%[label=\alph*]
  \item $h_1= 0$: this is a standard ARX model with one-step ahead predictor
    \begin{equation*}
      \hat{y}(k|k-1) = (1-A)y+Bu.
    \end{equation*}
  \item Here it is convenient to work with the scaled problem $y_C(k) = C^{-1}y(k)$ where the one-step ahead predictor is
    \begin{equation*}
      \hat{y}_C(k|k-1) = (1-A)y_C+Bu_C.
    \end{equation*}
    $C(z) = 1+z^{-1}$ is not stably invertible, so I am not sure this is really justified.

  \end{enumerate}

\item [\textbf{Part 2}] The model is
\begin{equation}
  \label{eq:prob1-eq3}
  y(k) = \underbrace{\frac{F(z)}{D(z)}}_{\doteq G_2(z)}u(k) + \frac{H_1(z)}{D(z)}e(k) + \frac{H_2(z)}{D(z)}w(k).
\end{equation}
The estimate $\hat{\theta}$ can be found using the one-step ahead predictor for the modified variables $y_H\doteq H_1^{-1}y$ and can be computed in the usual way: multiply by $D$ and rearrange it as
\begin{equation}
  \label{eq:prob1-eq3-linear}
  y = (1-D)y + Fu + H_1e + H_2w
\end{equation}
multiply by $H_1^{-1}$ (provided $H_1$ is stably invertible, which $H_1=1+z^{-1}$ is not)
\begin{equation*}
  y_H = (1-D)y_H + Fu_H + e + H_2w_H.
\end{equation*}
The one-step ahead predictor for $y_H$ is
\begin{align*}
  \hat{y}_H(k|k-1) &= (1-D)y_H + Fu_H + H_2w_H \\
                   &= \underbrace{\left[(1-D)y_H + Fu_H + z^{-1}w_H\right]}_{\Phi_H\theta_0} + w_H(k).
\end{align*}
Since $H_2$ has a pass-through term, the regressor becomes affine.

Multiplying $\hat{y}_H(k|k-1)$ by $H_1(z)$
\begin{equation*}
  H_1(z)\hat{y}_H(k|k-1) = (1-D)y + Fu + H_2w
\end{equation*}
does \emph{not} give the one step-ahead predictor for $y(k)$, because from the definition, we expect $y(k)-\hat{y}(k|k-1)=e(k)$; here instead we have
\begin{equation*}
  y(k)-H_1(z)\hat{y}_H(k|k-1) = H_1(z)e(k)
\end{equation*}
using eq.~\eqref{eq:prob1-eq3-linear}. To compute the correct one-step ahead predictor for $y(k)$, we use the formula derived in class
\begin{equation*}
  \hat{y} = H^{-1}Gx + (1-H^{-1})y, \hspace{1em} Gx = \frac{F}{D}u+\frac{H_2}{D}w, \hspace{1em} H = \frac{H_1}{D}.
\end{equation*}
The formula holds also when $\frac{H_2}{D}$ has a pass-through term (at least I do not see a reason why it should not). Straightforward application gives
\begin{align*}
  \hat{y}(k|k-1) &= Fu_H + H_2w_H + (H_1-D)y_H \\
                 &= (1-D)y_H + Fu_H + H_2w_H + (H_1-1)y_H
\end{align*}
where the second form can be used for the parametrization and the first to compute the prediction once the parameters $\hat{\theta}$ are known.
\end{enumerate}


%%%%%%%%%%%%%%%%%%%%%%%%%%%%%%%%%%%%%%%%%%%%%%%%%%%%%%%%%%%%
\section*{Problem 2}
\label{sec:prob2}

Given the two pulse responses $\{U_1,Y_1\}$ and $\{U_2,Y_2\}$ from the same system $G(z)$, how do we estimate $\hat{G}(z)$?

One way is to estimate the two separately and combine the estimates:
\begin{equation*}
  \hat{G}_1 = (\Phi_1^\top \Phi_1)^{-1}\Phi_1^\top Y_1,\hspace{2em} \hat{G}_2 = (\Phi_2^\top \Phi_2)^{-1}\Phi_2^\top Y_2, \hspace{2em}\hat{G} = \frac{\hat{G}_1 + \hat{G}_2}{2}
\end{equation*}
Another way is by stacking the measurements:
\begin{equation*}
  \begin{bmatrix}
    Y_1 \\ Y_2
  \end{bmatrix} =
  \begin{bmatrix}
    \Phi_1 \\ \Phi_2
  \end{bmatrix}\hat{g},\hspace{2em} \hat{g} = \frac{1}{\Phi_1^\top \Phi_1 + \Phi_2^\top \Phi_2}(\Phi_1^\top Y_1 + \Phi_2^\top Y_2)
\end{equation*}
Clearly the two results are not the same, even when the regressors are noise-free (\textit{e.g.} when one applies two different sets of control inputs) but they are both unbiased. Numerically, the standard deviation is also equal, see implementation \texttt{test3.m} but here I took the input signals to have the same amplitudes.

The decisive argument in favour of the second approach is that the measurement with the stronger signal should count more. This is encoded in $\Phi$: a stronger control input results in larger $\Phi$ and $Y$, since $Y = \Phi G_0 + e$.


%%%%%%%%%%%%%%%%%%%%%%%%%%%%%%%%%%%%%%%%%%%%%%%%%%%%%%%%%%%%
\section*{Problem 3}
\label{sec:prob3}

The systems model is
\begin{equation*}
  y(k) = \frac{B(z)}{A(z)}u(k) + \frac{1}{A(z)}e(k)
\end{equation*}
After computing $u(k)=r(k)-C(z)y(k)$, the problem is a standard ARX model.

\end{document}

%%% Local Variables:
%%% mode: latex
%%% TeX-master: t
%%% End:
